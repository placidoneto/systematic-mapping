%-[BEGIN]-----------------------------------------------------------------------
%This section contains the background necessary to develop the research.
Regarding our problem, two different lines of research were targeted:
\textit{(i)} data integration on a multi-cloud environment; and
\textit{(ii)} service level agreements and data integration.
%There are two different lines of research interesting to the proposal:

\subsection{Data integration on multi-cloud environment}
Correndo \textit{et al}~\cite{075} proposed an method for achieving RDF data integration
using SPARQL query rewrinting. 
The proposal takes into account other already existent approaches regarding rewriting 
the RDF graph pattern, focus on using data manipulation functions, in order to: (i) solve 
the entity co-reference problem which can occasion an ineffective data integration; 
and (ii) exploit ontology alignments with a particular interest in data manipulation. 
ElSheikh \textit{et al}~\cite{078} developed a system architecture which combines data integration,
service oriented architecture and distributed processing. The proposed system, called Service 
Oriented Data Integration based on MapReduce (SODIM), works on a pool of collaborative services and 
process a large number of databases represented as web services. 
Yau \textit{et al}~\cite{YauY08} presented a data integration approach focused on data privacy.
They propose a privacy preserving repository in order to integrate data from
different data sharing services. 
Based on users' integration requirements, the repository helps the process in retrieving and integrating
data across different services. Data security is supported on the repository and in the data sharing services as well.
Tian \textit{et al}~\cite{096} proposed an inter-cloud data integration system that is a trade-off between users' privacy requirements and the cost for those data protection and processing.
According to the users' privacy requirements, the query plan wrapper in the repository cloud will
create the users' query. This query will be then subdivided into sub-queries for
which corresponding service providers will be discovered. The sub-queries can
be then executed in the service provider or in the repository cloud.
Each option has it own charges for privacy and processing. 
Thus the query plan executor decides the best location to execute the sub-query
to meet privacy and cost constraints.

\subsection{Service level agreement and data integration}
Service level agreement (SLA) contracts have been widely adopted in the context of Cloud computing but still are topic of interest in the research community.
Research contributions mainly concern (i) SLA negotiation phase (step in which the
contracts are established between customers and providers) and (ii)
monitoring and allocation of cloud resources to detect and avoid SLA
violations.

In the domain of data integration approaches, Nie \textit{et al}~\cite{Nie07} proposed a data integration model guided by SLAs in a Grid environment.
Their architecture is subdivided into four parts: (i) a \textit{SLA-based
Resource Description Model} describes the database resources; (ii)
a \textit{SLA-based Query Model} normalizes the different queries based on the
SLA information; (iii) an \textit{SLA-based Matching Algorithm}  
selects the databases and finally (iv) a \textit{SLA-based Evaluation Model}
evaluates them in order to obtain the final query solution.
Considering our previous work~\cite{012}, to the best of our knowledge, we have
not identified more proposals concerning the use of SLAs combined with a data
integration approach in multi-cloud context.
