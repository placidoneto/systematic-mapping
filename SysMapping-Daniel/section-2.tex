%-[BEGIN]-----------------------------------------------------------------------
\section{Related Works}\label{sec:rw}
This section contains the background necessary to develop the research.
There are two different lines of research interesting to the proposal:
\textit{(i)} data integration on a multi-cloud environment; and
\textit{(ii)} service level agreements and data.

\subsection{Data integration on multi-cloud environment}
Correndo \textit{et al}~\cite{075} proposed an method for achieving RDF data integration
using SPARQL query rewrinting. 
The proposal takes into account other already existent approaches regarding rewriting 
the RDF graph pattern, focus on using data manipulation functions, in order to: (i) solve 
the entity co-reference problem which can occasion an ineffective data integration; 
and (ii) exploit ontology alignments with a particular interest in data manipulation. 
ElSheikh \textit{et al}~\cite{078} developed a system architecture which combines data integration,
service oriented architecture and distributed processing. The proposed system, called Service 
Oriented Data Integration based on MapReduce (SODIM), works on a pool of collaborative services and 
process a large number of data bases represented as web services. 
Yau \textit{et al}~\cite{YauY08} presented a data integration approach focus on data privacy.
Their work proposes a privacy preserving repository in order to integrate data from
different data sharing services. 
Based on users' integration requirements, the repository helps the process of retrieve and integrate
data across different services.
Their system supports the data secure on both sides: in the repository and in the data sharing services. 
Tian \textit{et al}~\cite{096} proposed an inter-cloud data integration system. 
Their system works as a mediator between users' privacy requirements and the cost for those data 
protection and processing.
According to the users' privacy requirements, the query plan wrapper in the repository cloud will
create the users' query. This query will be decomposed in sub-queries and for each query the 
corresponded service provider will be discovered. 
The sub-queries can be executed both: in the service provider or in the repository cloud.
Each option has it own charges for privacy and processing. 
Then, in order to find the better way between privacy and cost the query plan executor decides where 
the sub-query should be executed.

\subsection{Service level agreement and Data}
%-[END]-----------------------------------------------------------------------