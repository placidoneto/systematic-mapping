We applied the systematic mapping methodology presented in~\cite{SM:Petersen:2008} to our study on SLA-guided data integration on a multi-cloud environments. 
The proposed methodology consists in five steps (in which step has a result and the final one is the mapping):
\begin{description}
\item \textbf{Definition of research question} to define the \textit{research scope};
\item \textbf{Conduct search} in order to retrieve \textit{all candidate papers}. Those papers are selected applying a query which express the research interest to scientific databases;
\item  \textbf{Screening of papers} to select the \textit{relevant papers} to answer the research question based on a inclusion and exclusion criteria;
\item \textbf{Keywording using abstracts} to identify terms that helps on developing the \textit{classification scheme} (mapping categories to classify the papers); and
\item \textbf{Data extraction and mapping process} to sort the relevant papers into the mapping categories and produce the systematic mapping.
\end{description}
\bigskip The following subsections describes our first to fourth step in the mapping. The systematic mapping results are presented in the next section.

\subsection{Definition of research questions (RQs)}
The aim of this work is to identify in the literature how has \textit{SLA-guided data integration on a multi-cloud environments} been explored, discover possible gaps and the main results produced.
In order to achieve this goal we formulated three research questions:
\begin{description}
\item \textbf{Which are the SLA measures that have been applied most in the cloud?}
\item \textbf{How has the publication of papers on data integration involved towards cloud topics?}
\item \textbf{How and in which context have data integration guided by QoS models or requirements been explored in the literature?}
\end{description}

\subsection{Search and screening of papers}
\textcolor{red}{search string...}

\textcolor{red}{inclusion and exclusion criteria...}

Inclusion: The title and the abstract mention (i) approach using SLA which includes (or doesn't) data integration; (ii) any study regarding SLA in the context of cloud computing (I saw that there are some works on voip services. I believe this is out of our scope) or some improvement to SLA; (iii) data integration in the context of cloud computing; and (iv) QoS efforts regarding data integration;

A new version of our exclusion criteria:

Papers, books and others that the subject do not focus on the following topics must be excluded:
(i) an approach combining SLA, data integration and cloud and multi-cloud;
(ii) an approach using SLA contracts in the context of cloud and multi-cloud;
(iii) an approach applying some improvement to SLA;
(iv) a data integration approach in the context of cloud and multi-cloud; and
(v) a QoS approach regarding data integration.

\textcolor{red}{number  of results...}

\subsection{Keywording of abstracts}

\textcolor{red}{small introduction...facets...}

\begin{description}
\item \textbf{Data Integration Environment facet} represents the environment (architecture and deployment) in which data integration is being applied. The dimensions to this facet are: \textit{Cloud}, \textit{Data Warehouse}, \textit{Federated Database} and \textit{Multi-cloud}.
\begin{table}[h]
\caption{Caption here...}
\begin{center}
\begin{tabular}{|c|c|}
\hline 
\textbf{Dimension} & \textbf{Publication} \\ 
\hline 
Cloud & \cite{068,070,072,073,074,075,076,077,078,079,081,082,083,085} \\ 
      & \cite{087,088,089,090,094,095,096,097,098,099,100,102,103,105,106,107,108,109,110,113}\\ 
\hline 
Data Warehouse & \cite{066,091,114} \\ 
\hline 
Federated Database & \cite{071,089,112} \\ 
\hline 
Multi-cloud & \cite{012,071,093} \\ 
\hline 
\end{tabular}
\end{center}
\end{table}
\item \textbf{Data Integration Description facet} indicates the strategy used by authors in order to achieve data integration. The dimensions to this facet are: \textit{Knowledge}, \textit{Metadata} and \textit{Schema}.
\begin{table}[h]
\caption{Caption here...}
\begin{center}
\begin{tabular}{|c|c|}
\hline 
\textbf{Dimension} & \textbf{Publication} \\ 
\hline 
Knowledge & \cite{012,083} \\ 
\hline 
Metadata & \cite{066,108,113} \\ 
\hline 
Schema & \citep{070,071,072,073,075,083,089,091,102,112,114} \\ 
\hline 
\end{tabular}
\end{center}
\end{table}
\item \textbf{Data Quality facet} shows the data quality parameters applied in the publication. The dimensions are: \textit{Confidentiality}, \textit{Privacy}, \textit{Security}, \textit{SLA} (a publication is classified in this dimension when it does not focus on a specific quality parameter, but in general uses a SLA contract in order to specify one or more), \textit{Data Protection}, \textit{Data Provenance} and \textit{Others}.
\begin{table}[h]
\caption{Caption here...}
\begin{center}
\begin{tabular}{|c|c|}
\hline 
\textbf{Dimension} & \textbf{Publication} \\ 
\hline 
Confidentiality & \cite{024,104,109,111} \\ 
\hline 
Privacy & \cite{007,024,047,067,068,095,096,109,111,113} \\ 
\hline 
Security & \cite{065,081,093,109,112,113} \\ 
\hline 
SLA  & \cite{001,002,007,008,009,011,012,013,014,015,016,017,018,019,020,021,022,023,024,025,026} \\ 
     & \cite{027,028,029,030,031,032,033,034,035,036,037,038,039,040,041,042,043,044,045,046,047} \\
     & \cite{048,049,050,051,052,053,054,055,056,057,058,059,060,061,062,063,064,065} \\
\hline 
Data Protection & \cite{047,104,106} \\ 
\hline 
Data Provenance & \cite{012} \\ 
\hline 
Others & \cite{071,093,100} \\ 
\hline 
\end{tabular}
\end{center}
\end{table}
\item \textbf{SLA facet} is devote to present the main focus of how the SLA is used in the publication.  \textit{Language}, \textit{Model}, \textit{Resources} and \textit{Security}.
\begin{table}[h]
\caption{Caption here...}
\begin{center}
\begin{tabular}{|c|c|}
\hline 
\textbf{Dimension} & \textbf{Publication} \\ 
\hline 
Language & \cite{003,037,039,041,055,056,061} \\ 
\hline 
Model & \cite{001,002,003,005,006,007,008,009,010,012,013,014,015,016,017,018,019,020,021,022,023,024} \\ 
      & \cite{026,027,028,029,030,031,032,033,035,036,038,040,042,043,044,045,046,047,048,049,050,051} \\ 
      & \cite{053,054,055,057,058,059,060,061,063} \\ 
\hline 
Resources & \cite{053,064,110} \\ 
\hline 
Security & \cite{011,025,034,035,038,049,050,052,062,065,081,093,109,112,113} \\ 
\hline 
\end{tabular}
\end{center}
\end{table}
\item \textbf{Contribution facet} express the main publication contribution. 
The dimensions are \textit{Tool}, \textit{Literature Analysis}, \textit{Method}, \textit{Model}, \textit{Process} and \textit{Extended Study}.
\begin{table}[h]
\caption{Caption here...}
\begin{center}
\begin{tabular}{|c|c|}
\hline 
\textbf{Dimension} & \textbf{Publication} \\ 
\hline 
Tool & \cite{001,002,005,011,014,015,016,019,024,026,028,029,032,035,046,053,054} \\
     & \cite{056,061,064,065,066,068,070,071,074,077,078} \\ 
     & \cite{081,086,087,088,091,093,094,095,110,112,113} \\ 
\hline 
Literature Analysis & \cite{003,004,010,038,042,048,052,069,073,089,099,103,105,108,109,111} \\ 
\hline 
Method & \cite{011,043,051,075,076,092,101,102,106,107} \\ 
\hline 
Model & \cite{006,007,008,009,012,013,017,018,020,027,030,031,033,034,036,037} \\
      & \cite{039,040,041,044,045,049,050,055,056,057,058,059,060} \\
      & \cite{062,063,066,067,070,071,072,079,080,082,083,084,085,087,088} \\ 
      & \cite{090,096,098,114} \\  
\hline 
Process & \cite{021,022,023,025,096,100} \\ 
\hline 
Extended Study & \cite{047,097,104} \\ 
\hline 
\end{tabular}
\end{center}
\end{table}
\item \textbf{Research facet} is dedicated to classify in which kind of research the publication can be fitted in. The dimensions to this facet are: (i) \textit{Evaluation research} ; (ii) Validation research; (iii) Solution proposal; and (iv) Opinion.
\begin{table}[h]
\caption{Caption here...}
\begin{center}
\begin{tabular}{|c|c|}
\hline 
\textbf{Dimension} & \textbf{Publication} \\ 
\hline 
Evaluation research & \cite{008,026,048,069,073,074,089,094,099,102,103,105,111} \\ 
\hline 
Validation research & \cite{001,002,005,006,007,009,011,012,014,015,016,017,018,019,020,021,022,023,024} \\
         & \cite{025,027,028,029,030,031,032,033,034,036,037,039,045,051,057,059}\\ 
         & \cite{079,095,098,100,105,112} \\
\hline 
Solution proposal & \cite{008,035,038,040,041,042,043,044,046,047,048,049,050,051,053,054,055,056} \\ 
                  & \cite{058,060,061,062,063,064,065,066,067,068,070,071,072,075,076,077,078} \\
                  & \cite{079,080,081,082,083,084,085,086,087,088,090,091,092,093,094,095,096,097} \\ 
                  & \cite{098,100,101,102,104,106,107,110,113,114} \\
\hline 
Opinion paper & \cite{003,004,010,013,052,069,073,108,109} \\ 
\hline  
\end{tabular}
\end{center}
\end{table}
\end{description}