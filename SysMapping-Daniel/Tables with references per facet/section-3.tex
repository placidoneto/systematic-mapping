\section{Our Systematic mapping process}\label{sec:sm}

\textbf{Data Integration Environment facet} (See table~\ref{table:dimensions}). 
Represents the environment (architecture and deployment) in which data integration is being applied.
The dimensions to this facet are: Cloud, Data Warehouse, Federated Database and Multi-cloud.
\begin{table}[h]
\begin{center}
\begin{tabular}{p{4cm}p{10cm}}
\hline 
\textbf{Dimension} & \textbf{Publication} \\ 
\hline 
Cloud & 
\cite{106,110,105,107,108,109,068,070,072,113,073,074,075,076,077,078,079,081,082,083,085,087,088,089,090,094,095,096,097,098,099,100,102,103}\\ 
\hline 
Data Warehouse & \cite{066,114,091} \\ 
\hline 
Federated Database & \cite{071,089,112} \\ 
\hline 
Multi-cloud & \cite{012,071,093} \\ 
\hline 
\end{tabular}
\end{center}
\caption{Data Integration Environment facet}\label{table:dienviron}
\end{table}

\textbf{Data Integration Description facet} (See table~\ref{table:dimensions}).
Indicates the strategy used by authors in order to achieve data integration. 
The dimensions to this facet are: Knowledge, Metadata and Schema.
\begin{table}[h]
\begin{center}
\begin{tabular}{p{4cm}p{10cm}}
\hline 
\textbf{Dimension} & \textbf{Publication} \\ 
\hline 
Knowledge & \cite{012,083} \\ 
\hline 
Metadata & \cite{108,066,113} \\ 
\hline 
Schema & \cite{070,071,072,073,075,114,083,089,091,112,102} \\ 
\hline 
\end{tabular}
\end{center}
\caption{Data Integration Description facet}\label{table:didesc}
\end{table}

\textbf{Data Quality facet} (See table~\ref{table:dimensions}). 
Refers to data quality parameters applied in the publication. 
The dimensions are: Confidentiality, Privacy, Security, SLA, Data Protection, Data Provenance and Others.
Note that a publication is classified in the SLA dimension when it does not focus on a specific quality parameter, but in general uses a SLA contract in order to specify one or more.
\begin{table}[h]
\begin{center}
\begin{tabular}{p{4cm}p{10cm}}
\hline 
\textbf{Dimension} & \textbf{Publication} \\ 
\hline 
Confidentiality & \cite{104,109,111,024} \\ 
\hline 
Privacy & \cite{109,111,007,067,068,113,024,047,095,096} \\ 
\hline 
Security & \cite{109,113,081,093,112,065} \\ 
\hline 
SLA  &\cite{044,001,002,007,008,009,011,012,013,014,015,016,017,018,019,046,020,021,022,024,025,026,027,028,029,030,031,032,035,034,036,037,038,039,040,041,042,023,043,045,047,048,049,050,051,052,053,054,055,056,057,058,060,059,061,062,063,064,065,033}\\
\hline 
Data Protection & \cite{106,104,047} \\ 
\hline 
Data Provenance & \cite{012} \\ 
\hline 
Others & \cite{071,093,100} \\ 
\hline 
\end{tabular}
\end{center}
\caption{Data Quality facet}\label{table:dq}
\end{table}

\textbf{SLA facet} (See table~\ref{table:dimensions}).
This facet is devote to present how the SLA is mainly used in the publication. 
The dimension for this facet are: Language, Model, Resources and Security.
It is important to see that SLA appears as a dimension and as a facet.
As a facet, we are interest in the way SLA is used. 
As a dimension, it is just to indicate that the work
applies SLA in your solution.
\begin{table}[h]
\begin{center}
\begin{tabular}{p{4cm}p{10cm}}
\hline 
\textbf{Dimension} & \textbf{Publication} \\ 
\hline 
Language & \cite{003,037,039,041,055,056,061} \\ 
\hline 
Model & \cite{044,001,002,005,003,006,007,008,009,010,012,013,014,015,016,017,018,019,046,020,021,022,024,026,027,028,029,030,031,032,035,036,038,040,042,023,043,045,047,048,049,050,051,053,054,055,057,058,060,059,061,063,033}\\ 
\hline 
Resources & \cite{110,053,064} \\ 
\hline 
Security & \cite{109,011,113,025,035,034,081,038,049,050,052,093,062,112,065} \\ 
\hline 
\end{tabular}
\end{center}
\caption{SLA facet}\label{table:sla}
\end{table}

\textbf{Contribution facet} (See table~\ref{table:dimensions}).
Express the kind of contribution proposed by the author. 
The dimensions are Tool, Literature Analysis, Method, Model, Process and Extended Study.
\begin{table}[h]
\begin{center}
\begin{tabular}{p{4cm}p{10cm}}
\hline 
\textbf{Dimension} & \textbf{Publication} \\ 
\hline 
Tool & \cite{110,001,002,005,066,068,070,071,011,014,015,016,019,046,113,024,074,077,026,078,028,029,032,035,081,086,087,088,053,054,091,056,093,094,095,061,112,064,065}\\ 
\hline 
Literature Analysis & \cite{105,108,109,111,004,003,069,010,073,038,042,089,048,052,099,103} \\ 
\hline 
Method & \cite{106,107,011,075,076,043,051,092,101,102} \\ 
\hline 
Model & \cite{044,006,007,066,067,008,009,070,012,071,072,013,017,018,020,114,027,079,030,031,034,036,080,082,037,083,084,039,040,085,041,087,088,045,090,049,050,055,056,057,058,060,059,096,062,098,063,033}\\  
\hline 
Process & \cite{021,022,025,023,096,100} \\ 
\hline 
Extended Study & \cite{104,047,097} \\ 
\hline 
\end{tabular}
\end{center}
\caption{Contribution facet}\label{table:contribution}
\end{table}

\textbf{Research facet} (See table~\ref{table:dimensions}).
Dedicates to classify in which kind of research the publication can be fitted in. 
The dimensions to this facet are: Evaluation research, Validation research, Solution proposal 
and Opinion papers.

\begin{table}[!h]
\begin{center}
\begin{tabular}{p{4cm}p{10cm}}
\hline 
\textbf{Dimension} & \textbf{Publication} \\ 
\hline 
Evaluation research & \cite{008,026,048,069,073,074,089,094,099,102,103,105,111} \\ 
\hline 
Validation research & \cite{001,002,005,006,007,009,011,012,014,015,016,017,018,019,020,021,022,023,024,025,027,028,029,030,031,032,033,034,036,037,039,045,051,057,059,079,095,098,100,105,112}\\
\hline 
Solution proposal & \cite{008,035,038,040,041,042,043,044,046,047,048,049,050,051,053,054,055,056,058,060,061,062,063,064,065,066,067,068,070,071,072,075,076,077,078,079,080,081,082,083,084,085,086,087,088,090,091,092,093,094,095,096,097,098,100,101,102,104,106,107,110,113,114}\\
\hline 
Opinion paper & \cite{003,004,010,013,052,069,073,108,109} \\ 
\hline  
\end{tabular}
\end{center}
\caption{Research facet}\label{table:research}
\end{table}