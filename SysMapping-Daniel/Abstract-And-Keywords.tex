\begin{abstract}

The aim of this paper is to identify trends and open issues regarding the use of
SLA associated with data integration solutions on multi-cloud environments.
SLA demonstrated benefits for data analysis but has not been yet enough
considered on data integration. Therefore, we believe in the advantages of
SLA-based data integration in meeting better user requirement.   
To reach the target, we performed a Systematic Mapping \cite{SM:Petersen:2008} 
concerning the aforementioned topics in order to analyze the way in which they
are correlated. To do so, after a retrieve of scientific productions on the
subject, a classification of the results is done according to five facets: (i)
data integration environment (cloud; data warehouse; federated database;
multi-cloud); (ii) data integration description (knowledge; metadata; schema); (iii) data quality (confidentiality; privacy; security; SLA; data protection; data provenance; others); (iv) contribution (tool; literature analysis; model; method; process; extended study); and (v) research (evaluation; validation; opinion; solution). 
We combine the facets and the results have been expressed and analysed as bubble
charts. We ended the paper expressing our vision on SLA-guided data
integration in the multi-could context.

\end{abstract}

\keywords{Systematic Mapping, Service Level Agreement, Data Integration, Multi-cloud Environment.}