\section{Introduction}
\label{sec:intro}

%\idaniel{Doing Nadia's request, I've checked for works regarding our three keywords and also their variations and I didn't find anything. But I found one using Grid "SLA-Guided Data Integration on Database Grids".}

The emergence of new architectures like the cloud opens new opportunities to data processing. 
The possibility of having unlimited access to cloud resources and the ``pay as U go'' model make it possible to change the hypothesis for processing big  data collections.  Instead of designing processes and algorithms taking into consideration  limitations on resources availability, the cloud sets the focus on the economic cost implied of using resources and producing results by parallelizing their use while delivering data under subscription oriented cost models.
 
Integrating and processing heterogeneous Big Data, calls for efficient methods for correlating, associating, filtering them taking into consideration their ``structural'' characteristics (due to variety) but also their quality (veracity), e.g., trust, freshness, provenance, partial or total consistency. 
Existing data integration techniques have to be revisited considering weakly curated and modeled data sets. This can be done according to quality of service requirements expressed by their consumers and Service Level Agreement (SLA) contracts exported by the cloud providers that host  Big Data and deliver resources for executing the associated management processes. Yet, it is not an easy task to completely fulfill   SLA contracts particularly because they have to use several cloud providers to integrate the data they require under the conditions they expect.
Naturally, a collaboration between cloud providers becomes necessary~\cite{036} but this should be done in a user-friendly way, with high degree of transparency. As result, new challenges emerge: 
\begin{itemize}
\item Model SLA  in order to correlate the  properties that can be ensured by cloud providers with the expected requirements of their users;  
\item Big Data integration strategies across different cloud providers  to support final users tasks like decision making (e.g., the best itinerary to reach my job from home according to traffic situation), analysis and understanding; 
\item Compute and deliver query results efficiently according to users profiles and preferences (cloud subscription, minimizing the price and energy for accessing and exploiting data samples out of Big Data).
\end{itemize}

\subsection{Contribution}
This paper proposes an analysis of existing works fully or partially addressing the problem of integrating data in multi-cloud environments taking into consideration Service Level Agreement. Therefore, we apply the  methodology defined in~\cite{SM:Petersen:2008} called  systematic mapping  that enables to build a classification of the field. The classification consists of categories grouped into facets  to group and aggregate  publications according to frequencies. The study consists in  five interdependent tasks including (i) the definition of a research scope by defining research questions; (ii) retrieving candidate papers by querying different scientific databases (e.g. IEEE, Citeseer, DBLP); (iii) selecting relevant papers that can be used for answering the research questions by defining inclusion and exclusion criteria; (iv) defining a classification scheme by  analyzing the abstracts of selected papers to identify the terms that will be used as categories for classifying the papers; (v) producing a systematic mapping by sorting papers according to the classification scheme. 

Our final objective by applying the systematic mapping methodology is to identify trends and open issues regarding our research topic. Thus, we propose a systematic mapping consisting in three facets that classify existing scientific publications addressing  together or independently SLA, Data Integration in Multi-cloud environments. It shows the research trends of data integration as a result of the emergence of the cloud and the characteristics associated to Big Data that require ressources in order to be processed.


Our work addresses big data collections integration  in a multi-cloud hybrid context guided by user preferences statements and SLA contracts exported by different cloud providers. We propose an SLA guided continuous data provision and integration system exported as a DaaS by a cloud provider adapted to the vision of the economic model of the cloud such as accepting partial results delivered on demand or under predefined subscription models that can affect the quality of the results; accepting specific data duplication that can respect privacy but ensure data availability; accepting to launch a task that contributes to an integration on a first cloud whose SLA verifies a QoS require

\subsection{Organization of the paper}
The remainder of this paper is organized as follows. 
Section~\ref{sec:rw} discusses existing approaches studying data integration problems in multi-cloud contexts that take into account SLA contacts and that are supported by cloud providers for providing efficient solutions when the data collections they deal with are close to some Big Data properties: volume, variety, velocity, veracity, etc.
Section~\ref{sec:sm} describes our study of novel data integration perspectives and the evolution of the research works that address some aspects of the problem. Section~\ref{sec:qanalysis} gives a quantitative analysis of our study and identifies open issues in the field. Section \ref{sec:approach} gives the general lines of the approach we propose for guiding data integration using SLA agreements in a multi-cloud environment. We show initial experiments that show the feasibility of our approach.  
Section~\ref{sec:conc} concludes the paper and discusses future work. 


