\section{Introduction}
\label{sec:intro}

\idaniel{Doing Nadia's request, I've checked for works regarding our three keywords and also their variations and I didn't find anything. But I found one using Grid "SLA-Guided Data Integration on Database Grids".}

The emergence of new architectures like the cloud opens new opportunities to data processing. 
The possibility of having unlimited access to cloud resources and the ``pay as U go'' model make it possible to change the hypothesis for processing big  data collections. 
Instead of designing processes and algorithms taking into consideration  limitations on resources availability, the cloud sets the focus on the economic cost implied of using resources and producing results by parallelizing their use owhile delivering data under subscription oriented cost models.
 
Integrating and processing heterogeneous data collections, calls for efficient methods for correlating, associating, filtering them taking into consideration their ``structural'' characteristics (due to the different data models) but also their quality, e.g., trust, freshness, provenance, partial or total consistency. 
Existing data integration techniques have to be revisited considering weakly curated and modeled data sets. This can be done according to quality of service requirements expressed by their consumers and Service Level Agreement (SLA) contracts exported by the cloud providers that host  these collections and deliver resources for executing the associated management processes.

However, achieve all customer's quality of service is not an easy task for a single provider.
Naturally, a collaboration between cloud providers becomes necessary~\cite{036}.
As result to this new environment, new challenges rises.
Due to this, multi-cloud environments have been receiving more attention in the literature.

SLA have been widely discussed in the cloud computing context. ...

Data Integration...


Our work addresses big data collections integration  in a multi-cloud hybrid context guided by user preferences statements and SLA contracts exported by different cloud providers. The objective is to propose an SLA guided continuous data provision and integration system exported as a DaaS by a cloud provider adapted to the vision of the economic model of the cloud such as accepting partial results delivered on demand or under predefined subscription models that can affect the quality of the results; accepting specific data duplication that can respect privacy but ensure data availability; accepting to launch a task that contributes to an integration on a first cloud whose SLA verifies a QoS require



In this paper, we propose an analysis about the problem of integrating data in multi-cloud environments taking into consideration Service Level Agreement. 
The methodology defined in~\cite{SM:Petersen:2008} presents some guidelines to
performing a systematic mapping review in software engineering research
context. The systematic mapping is a defined method to build
a classification of a field of interest. The results analysis focuses on
frequencies of publications for categories (facets).  

The process workflow describes five interdependent tasks: \textit{(i)}
\textbf{definition of research question} to define the \textit{research scope}; \textit{(ii)} textbf{conduct search} in order to retrieve \textit{all candidate papers}. Those papers are selected applying a query which
express the research interest to scientific databases; \textit{(iii)}
\textbf{screening of papers} to select the \textit{relevant papers} to answer the research
question based on a inclusion and exclusion criteria; \textit{(iv)}
\textbf{keywording using abstracts} to identify terms that helps on developing the
\textit{classification scheme} (mapping categories to classify the papers); and
\textit{(v)} \textbf{fata extraction and mapping process} to sort the relevant
papers into the mapping categories and produce the systematic mapping.


This analysis was done using the Systematic Mapping Methodology~\cite{SM:Petersen:2008}. We propose three facets to determine the main issues to be addressed when SLA, Data Integration and Multi-cloud environments are put together in one research problem.

%The methodology consists in retrieving publications from scientific databases and classifying them in
%categories (called facets). As result we have charts which are used to answer specific research questions
%created in accordance with scientific interests. 


Our final objective is to identify trends and open issues regarding our research topic.

The remaining of this paper is organized as follows. 
The related work is in the section~\ref{sec:rw}. 
section~\ref{sec:sm} describes our steps regarding the methodology.
A quantitative analysis based on the facets is discussed in the section~\ref{sec:qanalysis} and conclusion and final
remarks comes in the section~\ref{sec:conc}. 