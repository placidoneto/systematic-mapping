\section{Introduction}
\label{sec:intro}

\idaniel{Doing Nadia's request, I've checked for works regarding our three keywords and also their variations and I didn't find anything. But I found one using Grid "SLA-Guided Data Integration on Database Grids".}

Cloud computing is a promising paradigm in which cloud providers can delivery infrastructure in 
a low-cost, efficient, flexible, on-demand and scalable way~\cite{014}.
In this context, it is provider's responsibility to guarantee customer's Quality
of Services (QoS) requirements (such as availability, efficiency, response time
and others).
Commonly, in order to specify these requirements and responsibilities, a Service
Level Agreement (SLA) contract should be agreed between both parts, the provider and customer~\cite{011}.

However, achieve all customer's quality of service is not an easy task for a single provider.
Naturally, a collaboration between cloud providers becomes necessary~\cite{036}.
As result to this new environment, new challenges rises.
Due to this, multi-cloud environments have been receiving more attention in the literature.

SLA have been widely discussed in the cloud computing context. ...

Data Integration...

In this paper, we propose an analysis about the problem of integrating data in multi-cloud environments taking into consideration Service Level Agreement. 
The methodology defined in~\cite{SM:Petersen:2008} presents some guidelines to
performing a systematic mapping review in software engineering research
context. The systematic mapping is a defined method to build
a classification of a field of interest. The results analysis focuses on
frequencies of publications for categories (facets).  

The process workflow describes five interdependent tasks: \textit{(i)}
\textbf{definition of research question} to define the \textit{research scope}; \textit{(ii)} textbf{conduct search} in order to retrieve \textit{all candidate papers}. Those papers are selected applying a query which
express the research interest to scientific databases; \textit{(iii)}
\textbf{screening of papers} to select the \textit{relevant papers} to answer the research
question based on a inclusion and exclusion criteria; \textit{(iv)}
\textbf{keywording using abstracts} to identify terms that helps on developing the
\textit{classification scheme} (mapping categories to classify the papers); and
\textit{(v)} \textbf{fata extraction and mapping process} to sort the relevant
papers into the mapping categories and produce the systematic mapping.


This analysis was done using the Systematic Mapping Methodology~\cite{SM:Petersen:2008}. We propose three facets to determine the main issues to be addressed when SLA, Data Integration and Multi-cloud environments are put together in one research problem.

%The methodology consists in retrieving publications from scientific databases and classifying them in
%categories (called facets). As result we have charts which are used to answer specific research questions
%created in accordance with scientific interests. 


Our final objective is to identify trends and open issues regarding our research topic.

The remaining of this paper is organized as follows. 
The related work is in the section~\ref{sec:rw}. 
section~\ref{sec:sm} describes our steps regarding the methodology.
A quantitative analysis based on the facets is discussed in the section~\ref{sec:qanalysis} and conclusion and final
remarks comes in the section~\ref{sec:conc}. 