\section{Introduction}
\label{sec:intro}

\idaniel{Doing Nadia's request, I've checked for works regarding our three keywords and also their variations and I didn't find anything. But I found one using Grid "SLA-Guided Data Integration on Database Grids".}
Cloud computing is a promising paradigm in which cloud providers can delivery infrastructure in 
a low-cost, efficient, flexible, on-demand and scalable way~\cite{014}.
In this context, it is provider's responsibility to guarantee customer's Quality
of Services (QoS) requirements (such as availability, efficiency, response time
and others).
Commonly, in order to specify these requirements and responsibilities, a Service
Level Agreement (SLA) contract should be agreed between both parts, the provider and customer~\cite{011}.

However, achieve all customer's quality of service is not an easy task for a single provider.
Naturally, a collaboration between cloud providers becomes necessary~\cite{036}.
As result to this new environment, new challenges rises.
Due to this, multi-cloud environments have been receiving more attention in the literature.

SLA have been widely discussed in the cloud computing context. ...

Data Integration...

In this paper, we performed a Systematic Mapping Methodology~\cite{SM:Petersen:2008} with the aim of
combining the three topics of interest discussed above: SLA, Data Integration and Multi-cloud environments.
The methodology consists in retrieving publications from scientific databases and classifying them in
categories (called facets). As result we have charts which are used to answer specific research questions
created in accordance with scientific interests. 
Our final objective is to identify trends and open issues regarding our research topic.

The remaining of this paper is organized as follows. 
The related work is in the section~\ref{sec:rw}. 
section~\ref{sec:sm} describes our steps regarding the methodology.
A quantitative analysis based on the facets is discussed in the section~\ref{sec:qanalysis} and conclusion and final
remarks comes in the section~\ref{sec:conc}. 