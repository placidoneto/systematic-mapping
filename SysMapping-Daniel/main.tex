\documentclass[preprint,12pt]{elsarticle}
\usepackage{geometry}
%\geometry{letterpaper}                   % ... or a4paper or a5paper or ...
\usepackage{graphicx}
\usepackage{xspace}
\usepackage{amssymb} 
\usepackage{epstopdf}
\usepackage{graphicx,color}

 
\usepackage{datatool}
\usepackage{tikz}
\usepackage{pgfplots}
\usepackage{pgfplotstable}
\usetikzlibrary{patterns}
\usepackage{lscape}
\usepackage{subfig}

%% Use the option review to obtain double line spacing
%% \documentclass[preprint,review,12pt]{elsarticle}
   
%% Use the options 1p,twocolumn; 3p; 3p,twocolumn; 5p; or 5p,twocolumn
%% for a journal layout: 
%% \documentclass[final,1p,times]{elsarticle}
%% \documentclass[final,1p,times,twocolumn]{elsarticle}
%% \documentclass[final,3p,times]{elsarticle}
%% \documentclass[final,3p,times,twocolumn]{elsarticle}
%% \documentclass[final,5p,times]{elsarticle}
%% \documentclass[final,5p,times,twocolumn]{elsarticle}

%% if you use PostScript figures in your article
%% use the graphics package for simple commands
%% \usepackage{graphics}
%% or use the graphicx package for more complicated commands
%% \usepackage{graphicx}
%% or use the epsfig package if you prefer to use the old commands
%% \usepackage{epsfig}

%% The amssymb package provides various useful mathematical symbols
\usepackage{amssymb}
%% The amsthm package provides extended theorem environments
%% \usepackage{amsthm}

%% The lineno packages adds line numbers. Start line numbering with
%% \begin{linenumbers}, end it with \end{linenumbers}. Or switch it on
%% for the whole article with \linenumbers after \end{frontmatter}.
%% \usepackage{lineno}

%% natbib.sty is loaded by default. However, natbib options can be
%% provided with \biboptions{...} command. Following options are
%% valid:

%%   round  -  round parentheses are used (default)
%%   square -  square brackets are used   [option]
%%   curly  -  curly braces are used      {option}
%%   angle  -  angle brackets are used    <option>
%%   semicolon  -  multiple citations separated by semi-colon
%%   colon  - same as semicolon, an earlier confusion
%%   comma  -  separated by comma
%%   numbers-  selects numerical citations
%%   super  -  numerical citations as superscripts
%%   sort   -  sorts multiple citations according to order in ref. list
%%   sort&compress   -  like sort, but also compresses numerical citations
%%   compress - compresses without sorting
%%
%% \biboptions{comma,round}

% \biboptions{}


\journal{Journal of Systems and Software}

%%% OUR MACROS %%%
\newcommand{\COMMENT}[1]{ }

%\usepackage[usenames,dvipsnames]{xcolor}
\usepackage{xcolor}


\usepackage{amsmath}
\usepackage[thmmarks,amsmath]{ntheorem}

\newcommand{\openbox}{\leavevmode
  \hbox to.77778em{%
  \hfil\vrule
  \vbox to.675em{\hrule width.6em\vfil\hrule}%
  \vrule\hfil}}

\theoremstyle{plain}
\theoremheaderfont{\normalfont\bfseries}
\theorembodyfont{\normalfont}
\theoremseparator{}
\theoremindent0cm
\theoremnumbering{arabic}
\newtheorem{algo}{Algorithm}

\theoremstyle{plain}
%\theoremheaderfont{\normalfont\itshape}
\theoremheaderfont{\normalfont\bfseries}
\theorembodyfont{\normalfont}
\theoremseparator{}
\theoremindent0cm
\theoremnumbering{arabic}
\theoremsymbol{\ensuremath{\openbox}} 
\newtheorem{example}{Example}


\theoremstyle{plain}
\theoremheaderfont{\normalfont\bfseries}
\theorembodyfont{\normalfont}
\theoremseparator{.}
\theoremindent0cm
\theoremnumbering{arabic}
\theoremsymbol{\ensuremath{\Box}} 
\newtheorem{defi}{Definition}

\theoremstyle{plain} 
\theoremsymbol{\ensuremath{\Box}} 
\theoremseparator{.} 
\newtheorem{prop}{Property}

\usepackage{listings}


\lstset{numbers=right, numbersep=5pt, numberstyle=\tiny, stepnumber=1,escapechar=\!,columns=fullflexible,
        morekeywords={procedure,let,for,do,if,then,else,add,choose,end,while,
        true,false,rise,exception,extend,resume,to,return,function}}

\begin{document}

\begin{frontmatter}

%% Title, authors and addresses

%% use the tnoteref command within \title for footnotes;
%% use the tnotetext command for the associated footnote;
%% use the fnref command within \author or \address for footnotes;
%% use the fntext command for the associated footnote;
%% use the corref command within \author for corresponding author footnotes;
%% use the cortext command for the associated footnote;
%% use the ead command for the email address,
%% and the form \ead[url] for the home page:
%%
%% \title{Title\tnoteref{label1}}
%% \tnotetext[label1]{}
%% \author{Name\corref{cor1}\fnref{label2}}
%% \ead{email address}
%% \ead[url]{home page}
%% \fntext[label2]{}
%% \cortext[cor1]{}
%% \address{Address\fnref{label3}}
%% \fntext[label3]{}

\title{SLA-based Data Integration on Multi-Cloud: A Systematic Mapping Analysis}

%% use optional labels to link authors explicitly to addresses:
%% \author[label1,label2]{<author name>}
%% \address[label1]{<address>}
%% \address[label2]{<address>}



\author[inst1]{Daniel Aguiar}
\author[inst2]{Nadia Bennani}
\author[inst1]{Chirine Ghedira}
\author[inst4]{Pl\'acido A. Souza Neto}
\author[inst5]{Genoveva Vargas-Solar}


%\address[inst4]{Universidad de las Am\'ericas-Puebla, LAFMIA -- Cholula, Mexico}
\address[inst1]{Universit\'e J-Moulin, Lyon 3 MAGELLAN, IAE -- France}
\address[inst2]{Universit\'e de Lyon, CNRS INSA-Lyon, LIRIS, UMR5205 -- France}
\address[inst4]{Instituto Federal do Rio Grande do Norte, Natal -- Brazil}
\address[inst5]{CNRS, LIG-LAFMIA, Saint Martin d'H\`eres -- France} 
  
\begin{abstract}

\ldots
\end{abstract}

\begin{keyword}
%% keywords here, in the form: keyword \sep keyword
\ldots \sep \ldots \sep Systematic Mapping.

%% MSC codes here, in the form: \MSC code \sep code
%% or \MSC[2008] code \sep code (2000 is the default)

\end{keyword}

\end{frontmatter}

%%
%% Start line numbering here if you want
%%
% \linenumbers

%% main text
%*********************************************************************************************************

%-[BEGIN]-----------------------------------------------------------------------
\section{Introduction}
\label{sec:intro}
%-[END]-----------------------------------------------------------------------

%-[BEGIN]-----------------------------------------------------------------------
\section{Related Works}

\subsection{Data integration on multi-cloud environment}

\subsection{Service level agreement}
%-[END]-----------------------------------------------------------------------

%-[BEGIN]-----------------------------------------------------------------------
\section{Systematic Mapping process}
We applied the systematic mapping methodology presented in~\cite{SM:Petersen:2008} to our study on SLA-guided data integration on a multi-cloud environments. 
The proposed methodology consists in five steps (in which step has a result and the final one is the mapping):
\begin{description}
\item \textbf{Definition of research question} to define the \textit{research scope};
\item \textbf{Conduct search} in order to retrieve \textit{all candidate papers}. Those papers are selected applying a query which express the research interest to scientific databases;
\item  \textbf{Screening of papers} to select the \textit{relevant papers} to answer the research question based on a inclusion and exclusion criteria;
\item \textbf{Keywording using abstracts} to identify terms that helps on developing the \textit{classification scheme} (mapping categories to classify the papers); and
\item \textbf{Data extraction and mapping process} to sort the relevant papers into the mapping categories and produce the systematic mapping.
\end{description}
\bigskip The following subsections describes our first to fourth step in the mapping. The systematic mapping results are presented in the next section.

\subsection{Definition of research questions (RQs)}
The aim of this work is to identify in the literature how has \textit{SLA-guided data integration on a multi-cloud environments} been explored, discover possible gaps and the main results produced.
In order to achieve this goal we formulated three research questions:
\begin{description}
\item \textbf{Which are the SLA measures that have been applied most in the cloud?}
\item \textbf{How has the publication of papers on data integration involved towards cloud topics?}
\item \textbf{How and in which context have data integration guided by QoS models or requirements been explored in the literature?}
\end{description}

\subsection{Search and screening of papers}
\textcolor{red}{search string...}

\textcolor{red}{inclusion and exclusion criteria...}

Inclusion: The title and the abstract mention (i) approach using SLA which includes (or doesn't) data integration; (ii) any study regarding SLA in the context of cloud computing (I saw that there are some works on voip services. I believe this is out of our scope) or some improvement to SLA; (iii) data integration in the context of cloud computing; and (iv) QoS efforts regarding data integration;

A new version of our exclusion criteria:

Papers, books and others that the subject do not focus on the following topics must be excluded:
(i) an approach combining SLA, data integration and cloud and multi-cloud;
(ii) an approach using SLA contracts in the context of cloud and multi-cloud;
(iii) an approach applying some improvement to SLA;
(iv) a data integration approach in the context of cloud and multi-cloud; and
(v) a QoS approach regarding data integration.

\textcolor{red}{number  of results...}

\subsection{Keywording of abstracts}

\textcolor{red}{small introduction...facets...}

\begin{description}
\item \textbf{Data Integration Environment facet} represents the environment (architecture and deployment) in which data integration is being applied. The dimensions to this facet are: \textit{Cloud}, \textit{Data Warehouse}, \textit{Federated Database} and \textit{Multi-cloud}.
\begin{table}[h]
\caption{Caption here...}
\begin{center}
\begin{tabular}{|c|c|}
\hline 
\textbf{Dimension} & \textbf{Publication} \\ 
\hline 
Cloud & • \\ 
\hline 
Data Warehouse & • \\ 
\hline 
Federated Database & • \\ 
\hline 
Multi-cloud & • \\ 
\hline 
\end{tabular}
\end{center}
\end{table}
\item \textbf{Data Integration Description facet} indicates the strategy used by authors in order to achieve data integration. The dimensions to this facet are: \textit{Knowledge}, \textit{Metadata} and \textit{Schema}.
\begin{table}[h]
\caption{Caption here...}
\begin{center}
\begin{tabular}{|c|c|}
\hline 
\textbf{Dimension} & \textbf{Publication} \\ 
\hline 
Knowledge & • \\ 
\hline 
Metadata & • \\ 
\hline 
Schema & • \\ 
\hline 
\end{tabular}
\end{center}
\end{table}
\item \textbf{Data Quality facet} shows the data quality parameters applied in the publication. The dimensions are: \textit{Confidentiality}, \textit{Privacy}, \textit{Security}, \textit{SLA} (a publication is classified in this dimension when it does not focus on a specific quality parameter, but in general uses a SLA contract in order to specify one or more), \textit{Data Protection}, \textit{Data Provenance} and \textit{Others}.
\begin{table}[h]
\caption{Caption here...}
\begin{center}
\begin{tabular}{|c|c|}
\hline 
\textbf{Dimension} & \textbf{Publication} \\ 
\hline 
Confidentiality & • \\ 
\hline 
Privacy & • \\ 
\hline 
Security & • \\ 
\hline 
SLA & • \\ 
\hline 
Data Protection & • \\ 
\hline 
Data Provenance & • \\ 
\hline 
Others & • \\ 
\hline 
\end{tabular}
\end{center}
\end{table}
\item \textbf{SLA facet} is devote to present the main focus of how the SLA is used in the publication.  \textit{Language}, \textit{Model}, \textit{Resources} and \textit{Security}.
\begin{table}[h]
\caption{Caption here...}
\begin{center}
\begin{tabular}{|c|c|}
\hline 
\textbf{Dimension} & \textbf{Publication} \\ 
\hline 
Language & • \\ 
\hline 
Model & • \\ 
\hline 
Resources & • \\ 
\hline 
Security & • \\ 
\hline 
\end{tabular}
\end{center}
\end{table}
\item \textbf{Contribution facet} express the main publication contribution. 
The dimensions are \textit{Tool}, \textit{Literature Analysis}, \textit{Method}, \textit{Model}, \textit{Process} and \textit{Extended Study}.
\begin{table}[h]
\caption{Caption here...}
\begin{center}
\begin{tabular}{|c|c|}
\hline 
\textbf{Dimension} & \textbf{Publication} \\ 
\hline 
Tool & • \\ 
\hline 
Literature Analysis & • \\ 
\hline 
Method & • \\ 
\hline 
Model & • \\ 
\hline 
Process & • \\ 
\hline 
Extended Study & • \\ 
\hline 
\end{tabular}
\end{center}
\end{table}
\item \textbf{Research facet} is dedicated to classify in which kind of research the publication can be fitted in. The dimensions to this facet are: (i) \textit{Evaluation research} ; (ii) Validation research; (iii) Solution proposal; and (iv) Opinion.
\begin{table}[h]
\caption{Caption here...}
\begin{center}
\begin{tabular}{|c|c|}
\hline 
\textbf{Dimension} & \textbf{Publication} \\ 
\hline 
Evaluation research & • \\ 
\hline 
Validation research & • \\ 
\hline 
Solution proposal & • \\ 
\hline 
Opinion paper & • \\ 
\hline  
\end{tabular}
\end{center}
\end{table}
\end{description}

%-[END]-----------------------------------------------------------------------

%-[BEGIN]-----------------------------------------------------------------------
\section{Mapping Analysis}
%-[END]-----------------------------------------------------------------------

%-[BEGIN]-----------------------------------------------------------------------
\section{Conclusion and final remarks}
 %-[END]-----------------------------------------------------------------------

%% References with bibTeX database:

\bibliographystyle{plain}
 \bibliography{bibliography}


\end{document}

%%
%% End of file `elsarticle-template-1a-num.tex'.
