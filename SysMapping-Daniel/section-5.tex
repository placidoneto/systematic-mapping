\textcolor{red}{Here goes the conclusion....}

%The mapping results presented can be the starting point to motivate new studies, support the investigation of specific problems not sufficiently explored yet. 
%The quantitative analysis provides an idea of the trends in service-based software devel- opment with NFR, including methodologies, languages and tools. 
%The distribution of the papers that deal with NFR shows that they are addressed in different domains but the vocabulary changes a lot and that there is a need of consensus, despite the existence of specifications like ISO/IEC 9126. 
%When NFR are addressed at the level of the services they are related to QoS measures like economy or economic cost, availability, authentication requirements for contacting a service. 
%NFR as defined by ISO/IEC 9126 are vast and papers address one or two at a time, particularly those re- lated to the software engineering domain. 
%Middleware solutions provide frameworks that consider different types of NFR but this concerns only the implementation stage of the software development process. This implies that the compliance between the design and the implementation might not be ensured.