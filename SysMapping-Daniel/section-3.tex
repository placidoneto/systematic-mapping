\section{Our Systematic mapping process}\label{sec:sm}

%\iplacido{I changed the word expression `step' to `task' in order to describe
%the methodology `workflow'.}  
%We applied the systematic mapping methodology presented
% in~\cite{SM:Petersen:2008} to our study on SLA-guided data integration on a multi-cloud environments.

 
% \begin{description}
% \item \textbf{Definition of research question} to define the \textit{research scope};
% \item \textbf{Conduct search} in order to retrieve \textit{all candidate papers}. Those papers are selected applying a query which express the research interest to scientific databases;
% \item  \textbf{Screening of papers} to select the \textit{relevant papers} to answer the research question based on a inclusion and exclusion criteria;
% \item \textbf{Keywording using abstracts} to identify terms that helps on developing the \textit{classification scheme} (mapping categories to classify the papers); and
% \item \textbf{Data extraction and mapping process} to sort the relevant papers into the mapping categories and produce the systematic mapping.
% \end{description}

% The following subsections describes our first to fourth step in the
% mapping. %The systematic mapping results are presented in the next section.   


%\iplacido{Defining references to the RQs.} 

%\subsection{Research questions (RQs)}
<<<<<<< HEAD
The aim of our bibliographic study using the systematic mapping methodology \cite{SM:Petersen:2008}  is to identify   the key contributions and the evolution of the research done on \textit{SLA-guided
data integration on a multi-cloud environments} and discover open issues and limitations of existing work.    
Our study is guided by  three research questions:

\paragraph{\textit{\textbf{RQ1:} Which are the SLA measures that have been mostly
applied  in the cloud?}} This question will help  to identify how SLA the type of properties used for characterizing and evaluating the services provided  by different clouds and the type of measures used for evaluating these properties.


\paragraph{ \textit{\textbf{RQ2:}  How have published papers on data
 integration evolved towards cloud topics?}} This question is devoted to identify the way  data integration problems addressed in the literature started  to include issues introduced by the cloud.

\paragraph{\textit{\textbf{RQ3:} In which way and in which context has data integration been linked to Quality of Service (QoS) measures in the literature?}} The objective of this question is to understand which QoS measures have been used for evaluating data integration and to determine the conditions in which  specific measures are particularly used.

%--------------------------------------------------------------------------------------------------------------------------------------------
=======
As mentioned before, the aim of our systematic mapping process is first to
check how \textit{SLA-guided data integration on a
multi-cloud environments} has been explored in the literature, and discover
possible gaps and issues.

The aim of our systematic mapping process is first to check how
\textit{SLA-guided data integration on a multi-cloud environments} has been
explored in the literature. Second, to discover possible gaps and possible
issues. To achieve this goal, we formulated three research questions:    

%In order to achieve this goal we formulated three research questions:

\paragraph{\textit{\textbf{RQ1:} Which are the SLA measures that have been
applied most in the cloud?}} This question will help us to identify how SLA
values have been applied in the cloud, and how they are been used.

\paragraph{ \textit{\textbf{RQ2:} How has the publication of papers on data
 integration involved towards cloud topics?}} This question is devoted to
 measure how publication on data integration are applied towards the cloud computing
environment.

\paragraph{\textit{\textbf{RQ3:} How and in which context have data integration
guided by QoS models or requirements been explored in the literature?}} The main
goal of this question is to analyse how the literature explores the use of
SLA-based data integration and in which context it has been applied.


%\paragraph{\textbf{RQ1:} Which are the SLA measures that have been applied most in
%the cloud?} 
%\paragraph{\textbf{RQ2:}  How has the publication of papers on data integration
%involved towards cloud topics?}
%\paragraph{\textbf{RQ3:} How and in which context have data integration guided by QoS
%models or requirements been explored in the literature?}

%\begin{enumerate}
%\item \textbf{RQ1:} Which are the SLA measures that have been applied most in
%the cloud?
%\item \textbf{RQ2:}  How has the publication of papers on data integration
%involved towards cloud topics?
%\item \textbf{RQ3:} How and in which context have data integration guided by QoS
%models or requirements been explored in the literature?
%\end{enumerate}

%\subsection{Search and selection of papers} \label{subsec:search}
>>>>>>> 3fe000b90f7808e9521b97f67fe76e07d16336ff
\subsection{Search and screening of papers} \label{subsec:search}
%--------------------------------------------------------------------------------------------------------------------------------------------

According to our research questions and our expertise in data integration we chose a set of keywords to define a complex query to be used for retrieving papers from four target publications databases: IEEE~\footnote{http://ieeexplore.ieee.org/},
ACM~\footnote{http://dl.acm.org/}, Science Direct~\footnote{http://www.sciencedirect.com/} and
CiteSeerX~\footnote{http://citeseerx.ist.psu.edu/}. We used the following conjunctive and disjunctive general query which was completed with associated terms from a tesaurus and rewritten according to the expression rules of advanced queries in each database: 


%As mentioned before, we are interested in merging three different topics: SLA, Data Integration and Multi-cloud environment.

%According to these keywords and correlated words the search query formulated was:
\medskip
%\begin{small}
%\begin{verbatim}
%(("SLA" OR "Service Level Agreement" OR "Service-Level Agreement") AND 
%   ("Cloud" OR "Multi-cloud" OR "Multi cloud" OR "Multicloud" OR "Inter-cloud" OR 
%      "Inter cloud" OR "Intercloud" OR "Federated cloud" OR "Cloud federation" OR 
%         "Hybrid cloud")) OR
%(("SLA" OR "Service Level Agreement" OR "Service-Level Agreement") AND 
%   ("Data Integration" OR "Data Integration Systems" OR "Sources Integration" OR 
%      "Multi Databases" OR "Multi-databases" OR "Multidatabases" OR 
%         "Distributed databases")) OR
%(("Data Integration" OR "Data Integration Systems" OR "Sources Integration" OR 
%   "Multi Databases" OR "Multi-databases" OR "Multidatabases" OR 
%      "Distributed databases") AND 
%   ("Cloud" OR "Multi-cloud" OR "Multi cloud" OR "Multicloud" OR "Inter-cloud" OR 
%      "Inter cloud" OR "Intercloud" OR "Federated cloud" OR "Cloud federation" OR 
%        "Hybrid cloud")) OR
%(("Data Integration" OR "Data Integration Systems" OR "Sources Integration" OR 
%   "Multi Databases" OR "Multi-databases" OR "Multidatabases" OR 
%      "Distributed databases") AND 
%   ("QoS" and "Quality of Service"))
%\end{verbatim}
%\end{small}
%\daniel{New format of the query. We saved some space.}

%\begin{small}
%\textit{(("SLA" OR "Service Level Agreement" OR "Service-Level Agreement") AND 
%   ("Cloud" OR "Multi-cloud" OR "Multi cloud" OR "Multicloud" OR "Inter-cloud" OR 
%      "Inter cloud" OR "Intercloud" OR "Federated cloud" OR "Cloud federation" OR 
%\medskip        "Hybrid cloud"))} \textbf{OR} \\ 
%\textit{(("SLA" OR "Service Level Agreement" OR "Service-Level Agreement") AND 
%   ("Data Integration" OR "Data Integration Systems" OR "Sources Integration" OR 
%      "Multi Databases" OR "Multi-databases" OR "Multidatabases" OR }
%\medskip        \textit{ "Distributed databases"))} \textbf{OR} \\
%\textit{(("Data Integration" OR "Data Integration Systems" OR "Sources Integration" OR 
%   "Multi Databases" OR "Multi-databases" OR "Multidatabases" OR 
%      "Distributed databases") AND 
%   ("Cloud" OR "Multi-cloud" OR "Multi cloud" OR "Multicloud" OR "Inter-cloud" OR 
%      "Inter cloud" OR "Intercloud" OR "Federated cloud" OR "Cloud federation" OR }
%\medskip       \textit{ "Hybrid cloud"))} \textbf{OR} \\
%\textit{(("Data Integration" OR "Data Integration Systems" OR "Sources Integration" OR 
%   "Multi Databases" OR "Multi-databases" OR "Multidatabases" OR 
%      "Distributed databases") AND 
%   ("QoS" and "Quality of Service"))}
%\end{small}
%\medskip

\begin{center}
\textit{("Service level agreement" AND "Data integration" AND "Multi-cloud environment")} \\ 
\end{center}
\medskip

Considering that each scientific database has different search engines, the general query was 
rewritten in several manners according to each database syntax.
%We searched and filtered relevant works in four steps.
%In the first step we searched in four scientific databases: IEEE~\footnote{http://ieeexplore.ieee.org/},
%ACM~\footnote{http://dl.acm.org/}, Science Direct~\footnote{http://www.sciencedirect.com/} and
These queries were applied to four databases: IEEE~\footnote{http://ieeexplore.ieee.org/},
ACM~\footnote{http://dl.acm.org/}, Science Direct~\footnote{http://www.sciencedirect.com/} and
CiteSeerX~\footnote{http://citeseerx.ist.psu.edu/}.
%We retrieved 1832 publications (See table~\ref{table:pub}).
As result, 1832 publications were retrieved (See table~\ref{table:pub}).

\begin{table}[!ht]
\begin{center}
\begin{tabular}{>{\centering\arraybackslash}p{2.5cm}|>{\centering\arraybackslash}p{2.5cm}|>{\centering\arraybackslash}p{2.5cm}|>{\centering\arraybackslash}p{2.5cm}}
\toprule
\textbf{Database} & \textbf{Amount} & \textbf{Included} & \textbf{Excluded} \\ 
\hline \toprule
\textbf{IEEE} & 658 & 56 & 602 \\ 
\hline 
\textbf{AMC} & 649 & 31 & 618	 \\ 
\hline 
\textbf{Science Direct} & 106 & 6 & 100 \\ 
\hline 
\textbf{CiteSeerX} & 419 & 21 & 398 \\ 
\hline 
\textit{Total} & 1832 & \textbf{114} & 1718 \\ 
\bottomrule \hline
\end{tabular} 
\end{center}
\caption{Number of papers retrieved in each scientific database}\label{table:pub}
\end{table}

We retrieved  a total of 1832 publications shown in Table~\ref{table:pub}. According to the systematic mapping methodology, the initial collection was cleaned and filtered according to inclusion and exclusion criteria applied as filters when analyzing the titles and abstracts of the papers (see table~\ref{table:criteria}).  In general, we only kept publications written in English, addressing SLA models and languages, quality measures, and/or (multi)-cloud topics related to data integration. As a result of the filtering process we excluded 1718 publications. 
The columns \textit{Included} and \textit{Excluded} in Table~\ref{table:pub} summarize the number of papers per database that were included and excluded for building the final collection with  114 publications.


\begin{table}[!htb]
\begin{center}
\begin{tabular}{p{10cm}}
\bottomrule \hline
\textbf{Inclusion criteria} \\ 
\hline 
- The text must be in English \\ 
- SLA approaches including data integration and/or multi-cloud environments\\
- Studies regarding SLA and cloud, describing models, languages and security issues \\
- Works describing improvements to SLA \\
- Data integration studies including cloud and/or multi-cloud  \\
- Quality of Service efforts regarding data integration \\
\bottomrule \hline 
\textbf{Exclusion criteria} \\ 
\hline 
- Publication with only power point version available \\ 
- SLA approaches regarding resource allocation \\
- Any paper out of the inclusion criteria  \\
\bottomrule \hline
\end{tabular} 
\end{center}
\caption{Inclusion and exclusion criteria}\label{table:criteria}
\end{table}

%--------------------------------------------------------------------------------------------------------------------------------------------
\subsection{Defining a classification scheme}
%--------------------------------------------------------------------------------------------------------------------------------------------

We analyzed the titles and abstracts of the papers of the corpus que derived in the previous phase using information retrieval techniques in order to identify the frequent relevant terms. We used this terms for building a classification scheme consisting of five facets the grouped frequent relevant terms. According to the systematic mapping methodology, the relevant frequent keywords are dimensions that represent subcategories within the facets that group them. We define the facets and dimensions of the classification scheme that we propose for studying SLA guided data integration in multi-cloud environments.
%According to our research interests, these papers were classified using the five facets. 
%The facets' dimensions were defined based on our knowledge and on the keywording process proposed by the
%methodology.  
%Each facet has its own dimensions. 
%These dimensions were proposed based on our knowledge and on the keywords found in the abstracts.   

%.  -   .  .  -   ..  -   ..  -   ..  -   ..  -   ..  -   ..  -   ..  -   ..  -   ..  -   ..  -   ..  -   ..  -   ..  -   ..  -   ..  -   ..  -   ..  -   ..  -   ..  -   ..  -   ..  -   ..  -   .  
\paragraph{Data Integration Environment}  
%.  -   .  .  -   ..  -   ..  -   ..  -   ..  -   ..  -   ..  -   ..  -   ..  -   ..  -   ..  -   ..  -   ..  -   ..  -   ..  -   ..  -   ..  -   ..  -   ..  -   ..  -   ..  -   ..  -   ..  -   .  
As shown in table~\ref{table:dienviron} this facet groups the dimensions that characterize the architectures used for delivering data integration services ({\em data warehouse} and  {\em federated database}) and  architectures used for deploying these services ({\em cloud} and {\em multi-cloud}).
%\begin{table}[h]
%\begin{center}
%\begin{tabular}{p{4cm}p{10cm}}
%\hline 
%\textbf{Dimension} & \textbf{Publication} \\ 
%\hline 
%Cloud & 
%\cite{106,110,105,107,108,109,068,070,072,113,073,074,075,076,077,078,079,081,082,083,085,087,088,089,090,094,095,096,097,098,099,100,102,103}\\ 
%\hline 
%Data Warehouse & \cite{066,114,091} \\ 
%\hline 
%Federated Database & \cite{071,089,112} \\ 
%\hline 
%Multi-cloud & \cite{012,071,093} \\ 
%\hline 
%\end{tabular}
%\end{center}
%\caption{Data Integration Environment facet}\label{table:dienviron}
%\end{table}

\begin{table}[!h]
\begin{center}
\begin{tabular}{p{4cm}p{4cm}}
\hline 
\textbf{Dimension} & \textbf{Publication} \\ 
\hline 
Cloud & 34 \\ 
\hline 
Data Warehouse & 3 \\ 
\hline 
Federated Database & 3 \\ 
\hline 
Multi-cloud & 3 \\ 
\hline 
\end{tabular}
\end{center}
\caption{Data Integration Environment facet:  papers per dimension}\label{table:dienviron}
\end{table}

%.  -   .  .  -   ..  -   ..  -   ..  -   ..  -   ..  -   ..  -   ..  -   ..  -   ..  -   ..  -   ..  -   ..  -   ..  -   ..  -   ..  -   ..  -   ..  -   ..  -   ..  -   ..  -   ..  -   ..  -   .  
\paragraph{Data Integration Description}
%.  -   .  .  -   ..  -   ..  -   ..  -   ..  -   ..  -   ..  -   ..  -   ..  -   ..  -   ..  -   ..  -   ..  -   ..  -   ..  -   ..  -   ..  -   ..  -   ..  -   ..  -   ..  -   ..  -   ..  -   .  
 As shown in table~\ref{table:didesc} groups the dimensions describing the type of data used for describing the databases content in order to  integrate them. Data integration can be done by using {\em meta-data, schema}, and {\em knowledge}.
%\begin{table}[h]
%\begin{center}
%\begin{tabular}{p{4cm}p{10cm}}
%\hline 
%\textbf{Dimension} & \textbf{Publication} \\ 
%\hline 
%Knowledge & \cite{012,083} \\ 
%\hline 
%Metadata & \cite{108,066,113} \\ 
%\hline 
%Schema & \cite{070,071,072,073,075,114,083,089,091,112,102} \\ 
%\hline 
%\end{tabular}
%\end{center}
%\caption{Data Integration Description facet}\label{table:didesc}
%\end{table}

\begin{table}[!h]
\begin{center}
\begin{tabular}{p{4cm}p{4cm}}
\hline 
\textbf{Dimension} & \textbf{Publication} \\ 
\hline 
Knowledge & 2 \\ 
\hline 
Metadata & 3 \\ 
\hline 
Schema & 11 \\ 
\hline 
\end{tabular}
\end{center}
\caption{Data Integration Description facet: papers per dimension}\label{table:didesc}
\end{table}

%.  -   .  .  -   ..  -   ..  -   ..  -   ..  -   ..  -   ..  -   ..  -   ..  -   ..  -   ..  -   ..  -   ..  -   ..  -   ..  -   ..  -   ..  -   ..  -   ..  -   ..  -   ..  -   ..  -   ..  -   .  
\paragraph{Data Quality} 
%.  -   .  .  -   ..  -   ..  -   ..  -   ..  -   ..  -   ..  -   ..  -   ..  -   ..  -   ..  -   ..  -   ..  -   ..  -   ..  -   ..  -   ..  -   ..  -   ..  -   ..  -   ..  -   ..  -   ..  -   .  
As shown in table~\ref{table:dq} this facet groups the dimensions that represent the parameters that can be used for measuring data quality. Measures can related directly to data like {\em confidentiality, privacy, security, protection and provenance} and to the conditions in which it is integrated and delivered like {\em SLA}.
%Note that a publication is classified in the SLA dimension when it does not focus on a specific quality parameter, but in general uses a SLA contract in order to specify one or more.
%\begin{table}[h]
%\begin{center}
%\begin{tabular}{p{4cm}p{10cm}}
%\hline 
%\textbf{Dimension} & \textbf{Publication} \\ 
%\hline 
%Confidentiality & \cite{104,109,111,024} \\ 
%\hline 
%Privacy & \cite{109,111,007,067,068,113,024,047,095,096} \\ 
%\hline 
%Security & \cite{109,113,081,093,112,065} \\ 
%\hline 
%SLA  &\cite{044,001,002,007,008,009,011,012,013,014,015,016,017,018,019,046,020,021,022,024,025,026,027,028,029,030,031,032,035,034,036,037,038,039,040,041,042,023,043,045,047,048,049,050,051,052,053,054,055,056,057,058,060,059,061,062,063,064,065,033}\\
%\hline 
%Data Protection & \cite{106,104,047} \\ 
%\hline 
%Data Provenance & \cite{012} \\ 
%\hline 
%Others & \cite{071,093,100} \\ 
%\hline 
%\end{tabular}
%\end{center}
%\caption{Data Quality facet}\label{table:dq}
%\end{table}

\begin{table}[!h]
\begin{center}
\begin{tabular}{p{4cm}p{4cm}}
\hline 
\textbf{Dimension} & \textbf{Publication} \\ 
\hline 
Confidentiality & 4 \\ 
\hline 
Privacy & 10 \\ 
\hline 
Security & 6 \\ 
\hline 
SLA  & 60\\
\hline 
Data Protection & 3 \\ 
\hline 
Data Provenance & 1 \\ 
\hline 
Others & 3 \\ 
\hline 
\end{tabular}
\end{center}
\caption{Data Quality: papers per dimension}\label{table:dq}
\end{table}

%.  -   .  .  -   ..  -   ..  -   ..  -   ..  -   ..  -   ..  -   ..  -   ..  -   ..  -   ..  -   ..  -   ..  -   ..  -   ..  -   ..  -   ..  -   ..  -   ..  -   ..  -   ..  -   ..  -   ..  -   .  
\paragraph{SLA expression}
%.  -   .  .  -   ..  -   ..  -   ..  -   ..  -   ..  -   ..  -   ..  -   ..  -   ..  -   ..  -   ..  -   ..  -   ..  -   ..  -   ..  -   ..  -   ..  -   ..  -   ..  -   ..  -   ..  -   ..  -   .  
 Service Level Agreement refers to the contracted delivery time of the service or performance it represents agreements between the user and a system expressed as a combination of weighted measures. 
As shown in Table~\ref{table:sla} this facet groups dimensions describing the way SLA is represented in order to associate it to service provision.
SLA can be expressed using a  {\em language},  {\em model}, {\em resources} concerned by SLA and degree of  {\em security} provided by a data integration service.

%\begin{table}[h]
%\begin{center}
%\begin{tabular}{p{4cm}p{10cm}}
%\hline 
%\textbf{Dimension} & \textbf{Publication} \\ 
%\hline 
%Language & \cite{003,037,039,041,055,056,061} \\ 
%\hline 
%Model & \cite{044,001,002,005,003,006,007,008,009,010,012,013,014,015,016,017,018,019,046,020,021,022,024,026,027,028,029,030,031,032,035,036,038,040,042,023,043,045,047,048,049,050,051,053,054,055,057,058,060,059,061,063,033}\\ 
%\hline 
%Resources & \cite{110,053,064} \\ 
%\hline 
%Security & \cite{109,011,113,025,035,034,081,038,049,050,052,093,062,112,065} \\ 
%\hline 
%\end{tabular}
%\end{center}
%\caption{SLA facet}\label{table:sla}
%\end{table}

\begin{table}[h]
\begin{center}
\begin{tabular}{p{4cm}p{4cm}}
\hline 
\textbf{Dimension} & \textbf{Publication} \\ 
\hline 
Language & 7 \\ 
\hline 
Model & 53 \\ 
\hline 
Resources & 3 \\ 
\hline 
Security & 15 \\ 
\hline 
\end{tabular}
\end{center}
\caption{SLA expression facet: papers per dimension}\label{table:sla}
\end{table}

%.  -   .  .  -   ..  -   ..  -   ..  -   ..  -   ..  -   ..  -   ..  -   ..  -   ..  -   ..  -   ..  -   ..  -   ..  -   ..  -   ..  -   ..  -   ..  -   ..  -   ..  -   ..  -   ..  -   ..  -   .  
\paragraph{Contribution} 
%.  -   .  .  -   ..  -   ..  -   ..  -   ..  -   ..  -   ..  -   ..  -   ..  -   ..  -   ..  -   ..  -   ..  -   ..  -   ..  -   ..  -   ..  -   ..  -   ..  -   ..  -   ..  -   ..  -   ..  -   .  
As shown in table~\ref{table:contribution} groups dimensions representing the kind of contribution proposed for addressing data integration. Contributions can concern abstractions like a
  {\em method}, a  {\em model}, or a {\em process}, software {\em tools} or bibliographic and systems analysis like {\em literature analysis} and {\em extended studies}.
%\begin{table}[h]
%\begin{center}
%\begin{tabular}{p{4cm}p{10cm}}
%\hline 
%\textbf{Dimension} & \textbf{Publication} \\ 
%\hline 
%Tool & \cite{110,001,002,005,066,068,070,071,011,014,015,016,019,046,113,024,074,077,026,078,028,029,032,035,081,086,087,088,053,054,091,056,093,094,095,061,112,064,065}\\ 
%\hline 
%Literature Analysis & \cite{105,108,109,111,004,003,069,010,073,038,042,089,048,052,099,103} \\ 
%\hline 
%Method & \cite{106,107,011,075,076,043,051,092,101,102} \\ 
%\hline 
%Model & \cite{044,006,007,066,067,008,009,070,012,071,072,013,017,018,020,114,027,079,030,031,034,036,080,082,037,083,084,039,040,085,041,087,088,045,090,049,050,055,056,057,058,060,059,096,062,098,063,033}\\  
%\hline 
%Process & \cite{021,022,025,023,096,100} \\ 
%\hline 
%Extended Study & \cite{104,047,097} \\ 
%\hline 
%\end{tabular}
%\end{center}
%\caption{Contribution facet}\label{table:contribution}
%\end{table}

\begin{table}[h]
\begin{center}
\begin{tabular}{p{4cm}p{4cm}}
\hline 
\textbf{Dimension} & \textbf{Publication} \\ 
\hline 
Tool & 39\\ 
\hline 
Literature Analysis & 16 \\ 
\hline 
Method & 10 \\ 
\hline 
Model & 48 \\  
\hline 
Process & 6 \\ 
\hline 
Extended Study & 3 \\ 
\hline 
\end{tabular}
\end{center}
\caption{Contribution facet: papers per dimension}\label{table:contribution}
\end{table}

%.  -   .  .  -   ..  -   ..  -   ..  -   ..  -   ..  -   ..  -   ..  -   ..  -   ..  -   ..  -   ..  -   ..  -   ..  -   ..  -   ..  -   ..  -   ..  -   ..  -   ..  -   ..  -   ..  -   ..  -   .  
\paragraph{Paper type}
%.  -   .  .  -   ..  -   ..  -   ..  -   ..  -   ..  -   ..  -   ..  -   ..  -   ..  -   ..  -   ..  -   ..  -   ..  -   ..  -   ..  -   ..  -   ..  -   ..  -   ..  -   ..  -   ..  -   ..  -   .  
As shown in   Table~\ref{table:research} this facet groups dimensions representing the type and purpose of the papers that address data integration: {\em evaluation, validation, solution} and {\em  position}.

%\begin{table}[h]
%\begin{center}
%\begin{tabular}{p{4cm}p{10cm}}
%\hline 
%\textbf{Dimension} & \textbf{Publication} \\ 
%\hline 
%Evaluation research & \cite{008,026,048,069,073,074,089,094,099,102,103,105,111} \\ 
%\hline 
%Validation research & \cite{001,002,005,006,007,009,011,012,014,015,016,017,018,019,020,021,022,023,024,025,027,028,029,030,031,032,033,034,036,037,039,045,051,057,059,079,095,098,100,105,112}\\
%\hline 
%Solution proposal & \cite{008,035,038,040,041,042,043,044,046,047,048,049,050,051,053,054,055,056,058,060,061,062,063,064,065,066,067,068,070,071,072,075,076,077,078,079,080,081,082,083,084,085,086,087,088,090,091,092,093,094,095,096,097,098,100,101,102,104,106,107,110,113,114}\\
%\hline 
%Opinion paper & \cite{003,004,010,013,052,069,073,108,109} \\ 
%\hline  
%\end{tabular}
%\end{center}
%\caption{Research facet}\label{table:research}
%\end{table}

\begin{table}[h]
\begin{center}
\begin{tabular}{p{4cm}p{4cm}}
\hline 
\textbf{Dimension} & \textbf{Publication} \\ 
\hline 
Evaluation research & 13 \\ 
\hline 
Validation research & 41 \\
\hline 
Solution proposal & 63\\
\hline 
Opinion paper & 9 \\ 
\hline  
\end{tabular}
\end{center}
\caption{Paper type facet: papers per dimension}\label{table:research}
\end{table}

%\idaniel{New table...}
%\begin{table}[!h]
%\begin{center}
%\begin{tabular}{p{4cm}p{10cm}}
%\hline 
%\textbf{Dimension} & \textbf{Publication} \\ 
%\hline 
%Cloud & 
%\cite{106,110,105,107,108,109,068,070,072,113,073,074,075,076,077,078,079,081,082,083,085,087,088,089,090,094,095,096,097,098,099,100,102,103}\\ 
%\hline 
%Data Warehouse & \cite{066,114,091} \\ 
%\hline 
%Federated Database & \cite{071,089,112} \\ 
%\hline 
%Multi-cloud & \cite{012,071,093} \\ 
%\hline 
%\hline 
%Knowledge & \cite{012,083} \\ 
%\hline 
%Metadata & \cite{108,066,113} \\ 
%\hline 
%Schema & \cite{070,071,072,073,075,114,083,089,091,112,102} \\ 
%\hline 
%\hline 
%Confidentiality & \cite{104,109,111,024} \\ 
%\hline 
%Privacy & \cite{109,111,007,067,068,113,024,047,095,096} \\ 
%\hline 
%Security & \cite{109,113,081,093,112,065} \\ 
%\hline 
%SLA  &\cite{044,001,002,007,008,009,011,012,013,014,015,016,017,018,019,046,020,021,022,024,025,026,027,028,029,030,031,032,035,034,036,037,038,039,040,041,042,023,043,045,047,048,049,050,051,052,053,054,055,056,057,058,060,059,061,062,063,064,065,033}\\
%\hline 
%Data Protection & \cite{106,104,047} \\ 
%\hline  
%Data Provenance & \cite{012} \\ 
%\hline 
%Others & \cite{071,093,100} \\ 
%\hline 
%\hline 
%Language & \cite{003,037,039,041,055,056,061} \\ 
%\hline 
%Model & \cite{044,001,002,005,003,006,007,008,009,010,012,013,014,015,016,017,018,019,046,020,021,022,024,026,027,028,029,030,031,032,035,036,038,040,042,023,043,045,047,048,049,050,051,053,054,055,057,058,060,059,061,063,033}\\ 
%\hline 
%Resources & \cite{110,053,064} \\ 
%\hline 
%Security & \cite{109,011,113,025,035,034,081,038,049,050,052,093,062,112,065} \\ 
%\hline
%\hline 
%Tool & \cite{110,001,002,005,066,068,070,071,011,014,015,016,019,046,113,024,074,077,026,078,028,029,032,035,081,086,087,088,053,054,091,056,093,094,095,061,112,064,065}\\ 
%\hline 
%Literature Analysis & \cite{105,108,109,111,004,003,069,010,073,038,042,089,048,052,099,103} \\ 
%\hline 
%Method & \cite{106,107,011,075,076,043,051,092,101,102} \\ 
%\hline 
%Model & \cite{044,006,007,066,067,008,009,070,012,071,072,013,017,018,020,114,027,079,030,031,034,036,080,082,037,083,084,039,040,085,041,087,088,045,090,049,050,055,056,057,058,060,059,096,062,098,063,033}\\  
%\hline 
%Process & \cite{021,022,025,023,096,100} \\ 
%\hline 
%Extended Study & \cite{104,047,097} \\ 
%\hline 
%\hline 
%Evaluation research & \cite{008,026,048,069,073,074,089,094,099,102,103,105,111} \\ 
%\hline 
%Validation research & \cite{001,002,005,006,007,009,011,012,014,015,016,017,018,019,020,021,022,023,024,025,027,028,029,030,031,032,033,034,036,037,039,045,051,057,059,079,095,098,100,105,112}\\
%\hline 
%Solution proposal & \cite{008,035,038,040,041,042,043,044,046,047,048,049,050,051,053,054,055,056,058,060,061,062,063,064,065,066,067,068,070,071,072,075,076,077,078,079,080,081,082,083,084,085,086,087,088,090,091,092,093,094,095,096,097,098,100,101,102,104,106,107,110,113,114}\\
%\hline 
%Opinion paper & \cite{003,004,010,013,052,069,073,108,109} \\ 
%\hline  
%\end{tabular}
%\end{center}
%\caption{List of publications per dimension}\label{table:dimensions}

%\end{table}

Facets define the classification scheme we used for  classifying the papers in the corpus according to dimensions. 
Each paper can be classified into one or several dimensions of each facet. 
%Tables \ref{table:dienviron} to \ref{table:research} summarize the classification results.
