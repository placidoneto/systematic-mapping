The emergence of new architectures like the cloud opens new opportunities for data integration. 
The possibility of having unlimited access to cloud resources and the ``pay as U go'' model make it possible to change the hypothesis for processing big  data collections.  Instead of designing processes and algorithms taking into consideration  limitations on resources availability, the cloud sets the focus on the economic cost implied when using resources and producing results by parallelizing their use while delivering data under subscription oriented cost models.
 
Integrating and processing heterogeneous huge data collections (i.e., Big Data) calls for efficient methods for correlating, associating, filtering them taking into consideration their ``structural'' characteristics (due to data variety) but also their quality (veracity), e.g., trust, freshness, provenance, partial or total consistency. 
Existing data integration techniques need to be revisited considering weakly curated and modeled data sets provided by different services under different quality conditions. Data integration can be done according to  (i) quality of service (QoS) requirements expressed by their consumers and (ii) Service Level Agreements (SLA)  exported by the cloud providers that host  huge data collections and deliver resources for executing the associated management processes. 
Yet, it is not an easy task to completely enforce SLAs particularly because
consumers use several cloud providers to store, integrate and process the data
they require under the specific conditions they expect. A major concern when
integrating data from different sources (services) is privacy that can be
associated to the conditions in which integrated data collections are built and
share ~\cite{YauY08}.     
Naturally, a collaboration between cloud providers becomes necessary~\cite{036}
but this should be also done in a user-friendly way, with some degree of
transparency. On the other hand, the definition of a SLA extension seems to be
a reasonable way to better guide data integration on a (multi-)cloud
environment.

%


%Particularly, for queries that call several services deployed  on different clouds.
 
%\subsection{Contribution}
% Data integration has to be revisited according to  the properties of data
% collections (volume, variety, velocity), to service oriented data provision
% contexts normally done in the context of cloud environments. Such environments
% (multi-cloud)  provide the required storage, computing and processing resources
% but they call for new strategies for integrating data considering different
% SLAs, subscription conditions, and data consumption requirements and
% preferences. 

In this context, the contribution of our work is
proposes a classification scheme of existing works fully or partially addressing
the problem of integrating data in multi-cloud environments taking into
consideration an extended form of Service Level Agreement. 
 

% Second, the description of our vision
% for guiding data integration in multi-cloud environments  by SLA and data
% consumers preferences.

The classification scheme results from  applying the  methodology defined
in~\cite{SM:Petersen:2008} called  \textit{systematic mapping}  for defining a
classification of a field. A classification consists of categories clustered
into facets in which publications (i.e., papers) are aggregated according to
frequencies (i.e., number of published papers). According to the methodology,
the study consists in  five interdependent steps including (i) the definition of
a research scope by defining research questions; (ii) retrieving candidate
papers by querying different scientific databases (e.g. IEEE, Citeseer, DBLP);
(iii) selecting relevant papers that can be used for answering the research
questions by defining inclusion and exclusion criteria; (iv) defining a
classification scheme by analyzing the abstracts of the selected papers to
identify the terms that will be used as categories for classifying the papers;
(v) producing a systematic mapping by sorting papers according to the
classification scheme.              

% Our final objective by applying the systematic mapping methodology is to
% identify trends and open issues regarding data integration in multi-cloud
% environments. Thus, our classification scheme consists in four facets that
% classify existing scientific publications addressing  together or independently
% SLA, Data Integration in Multi-cloud environments. We define two additional
% facets to identify the type of papers (e.g., position, survey, etc) and the type
% of contribution (e.g., model, architecture, system). It shows the research
% trends of data integration as a result of the emergence of the cloud and the
% characteristics associated to huge data collections processing. The 
% classification scheme that we propose supports the proposal of our vision for 
% proposing an original data integration solution according to current trends in
% the area.    

%\subsection{Organization of the paper}
The remainder of this paper is organized as follows. Section~\ref{sec:sm}
describes our study of data integration perspectives and the evolution of the
research works that address some aspects of the problem. It gives a quantitative
analysis of our study and identifies open issues in the field.
Section~\ref{sec:conc} concludes the paper and discusses future work with
reference to the stated problem.


